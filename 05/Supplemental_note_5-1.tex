\documentclass[a4paper,11pt]{article}
\usepackage{a4wide}
\usepackage{amsmath}
\usepackage{amssymb}
\begin{document}
\begin{center}
  {\LARGE\bf Supplemental note for Week 5 Part 1}
  \end{center}
\begin{flushright}
  {\large\bf ver. 20170417-01}\\
 \ \\
{\large\bf Ryoichi Yamamoto}\\
\end{flushright}

\section{Langevin Eq. $\rightarrow$ Fokker-Planck Eq.}

Let us start with the Langevin equation for a Brownian particle
\begin{equation}
m\frac{d\mathbf{V}(t)}{dt}=-\zeta\mathbf{V}(t)+\mathbf{F}(t),\tag{F2}
\end{equation}
where the random force satisfies
\begin{equation}
\langle\mathbf{F}(t)\rangle=\mathbf{0}\tag{F3}
\end{equation}
\begin{equation}
\langle\mathbf{F}(t)\mathbf{F}(0)\rangle=2k_BT\zeta\mathbf{I}\delta(t).\tag{F4}
\end{equation}
To derive a useful partial differential equation for the probability
distribution functions, we redefine the
Langevin equation in the following 1-dimensional form, assuming that the particle is subject to a conservative force $F_p(R(t),t)=-\frac{\partial U}{\partial R}$ due to a potential $U(R(t),t)$,
\begin{eqnarray}
m\frac{dV(t)}{dt}&=&-\zeta V(t)+F(t)+F_{p}(R(t),t),
\end{eqnarray}
where $R(t)$ represents the position of the Brownian particle at time $t$.
\\
Hereafter, we neglect the inertial term $m\frac{dV}{dt}\simeq 0$, since
we are mainly interested in the long-time behavior
$t\gg\frac{\zeta}{m}$ of the Brownian particle.
We thus obtain the overdamped Langevin equation,
\begin{eqnarray}
0&=&-\zeta V(t)+F(t)+F_{p}(R(t),t)\\
V(t)=\frac{dR(t)}{dt}&=&\frac{F_p(R(t),t)}{\zeta}+\frac{F(t)}{\zeta}.
\end{eqnarray}
Let us now rewrite this equation (3) in the following more general
form (making the substitution $R(t)\longrightarrow r(t)$)
\begin{equation}
\frac{dr(t)}{dt}=f(r(t),t)+g\eta(t),
\end{equation}
where $f$ is an arbitrary function of $r(t)$ and $t$, $g$ is a
constant, and $\eta(t)$ is a Gaussian white noise random variable with
\begin{eqnarray}
\langle\eta(t)\rangle&=&0\\
\langle\eta(t)\eta(t')\rangle&=&\delta(t-t').
\end{eqnarray}
Integration of Eq.(4) from $t=t_i$ to $t_{i+1}=t_i+\Delta t$, yields 
\begin{eqnarray}
r(t_{i+1})-r(t_i)&=& \int_{t_i}^{t_{i+1}}f(r(t),t)dt + g\int_{t_i}^{t_{i+1}}\eta(t)dt\\
&\simeq& f(r(t_i),t_i)\Delta t + g\sqrt{\Delta t}\tilde{\eta}(t_i),
\end{eqnarray}
with
\begin{eqnarray}
\langle\tilde{\eta}(t)\rangle&=&0\\
\langle\tilde{\eta}(t_i)\tilde{\eta}(t_j)\rangle&=&\delta_{ij}
\end{eqnarray}
(see Eq.(F11) and the supplemental note for Week 4 Part 1).
\\
We now consider a general function $p\left[r(t)\right]$ of $r(t)$ and
make a Taylor expansion of it in terms of $\Delta t$, keeping only the
terms equal to and lower than the first-order in $\Delta t$
\begin{eqnarray}
p\left[r(t_{i+1})\right]&=&p\left[r(t_i)+f(r(t_i),t_i)\Delta t + g\sqrt{\Delta t}\tilde{\eta}(t_i)\right]\\
&\simeq&
p\left[r(t_{i})\right]+\dot{p}\left[r(t_{i})\right]
\left(f(r(t_i),t_i)\Delta t+g\sqrt{\Delta t}\tilde{\eta}(t_i)\right)\nonumber\\
&&+\frac{\ddot{p}\left[r(t_{i})\right]}{2}\left(f(r(t_i),t_i)\Delta t+g\sqrt{\Delta t}\tilde{\eta}(t_i)\right)^2+\cdots\\
&\simeq&
p\left[r(t_{i})\right]
+\dot{p}\left[r(t_{i})\right]g\tilde{\eta}(t_i)\Delta t^{0.5}
+\dot{p}\left[r(t_{i})\right]f(r(t_i),t_i)\Delta t\nonumber\\
&&+\frac{\ddot{p}\left[r(t_{i})\right]}{2}g^2\tilde{\eta}^2(t_i)\Delta t+{\cal O}(\Delta t^{1.5}),
\end{eqnarray}
where $\dot{p}\left[r(t_{i})\right]\equiv\left.\frac{dp}{dr}\right|_{r=r(t_{i})}$ and
$\ddot{p}\left[r(t_{i})\right]\equiv\left.\frac{d^2p}{dr^2}\right|_{r=r(t_{i})}$.
\\
Moving $p\left[r(t_{i})\right]$ to the left-hand-side, dividing by
$\Delta t$, and taking a statistical average, we obtain 
\begin{eqnarray}
\left\langle\frac{p\left[r(t_{i+1})\right]-p\left[r(t_i)\right]}{\Delta t}\right\rangle
&=&
\left\langle \dot{p}\left[r(t_{i})\right]g\tilde{\eta}(t_i)\Delta t^{-0.5}\right\rangle
+\left\langle \dot{p}\left[r(t_{i})\right]f(r(t_i),t_i)\right\rangle\nonumber\\
&&+\left\langle\frac{\ddot{p}\left[r(t_{i})\right]}{2}g^2\tilde{\eta}^2(t_i)\right\rangle\\
&=&
\left\langle \dot{p}\left[r(t_{i})\right]\right\rangle g\left\langle\tilde{\eta}(t_i)\right\rangle\Delta t^{-0.5}
+\left\langle \dot{p}\left[r(t_{i})\right]f(r(t_i),t_i)\right\rangle\nonumber\\
&&+\left\langle\frac{\ddot{p}\left[r(t_{i})\right]}{2}g^2\right\rangle\left\langle\tilde{\eta}^2(t_i)\right\rangle
\end{eqnarray}
Substituting $\langle\tilde{\eta}(t_i)\rangle=0$ and $\langle\tilde{\eta}^2(t_i)\rangle=1$, and then taking the limit of $\Delta t\rightarrow 0$, one finally obtains
\begin{equation}
  \left\langle\frac{dp(r(t))}{dt}\right\rangle
  =\left\langle \dot{p}(r(t)f(r(t),t)\right\rangle+\left\langle\frac{\ddot{p}(r(t))}{2}g^2\right\rangle
\end{equation}
\\
Using the definition of a statistical average in terms of the
probability distribution function $P(r,t)$, we have
\begin{equation}
\left\langle p(r(t))\right\rangle=\int P(r,t)p(r)dr.
\end{equation}
The corresponding expressions for each of the terms appearing in
Eq.(16) are
\begin{eqnarray}
\left\langle \frac{dp(r(t))}{dt}\right\rangle&=&\frac{\partial}{\partial t}\int P(r,t)p(r)dr\\
\left\langle \dot{p}(r(t)f(r(t),t)\right\rangle&=&\int P(r,t)\dot{p}(r)f(r,t)dr\\
\left\langle \frac{\ddot{p}(r(t))}{2}g^2\right\rangle&=&\int P(r,t)\frac{\ddot{p}(r)}{2}g^2dr.
\end{eqnarray}
In the case that 
$p(r(t))=\delta(r(t)-R)$, the above averages can be calculated as follows,
\begin{eqnarray}
\left\langle \frac{dp(r(t))}{dt}\right\rangle&=&\frac{\partial}{\partial t}\int P(r,t)\delta(r-R)dr=\frac{\partial}{\partial t}P(R,t)\\
\left\langle \dot{p}(r(t)f(r(t),t)\right\rangle&=&\int P(r,t)\frac{d}{dr}\delta(r-R)f(r,t)dr\\
&=&-\int \frac{d}{dr}\left(P(r,t)f(r,t)\right)\delta(r-R)dr\\
&=&-\frac{\partial}{\partial  R}\left(P(R,t)f(R,t)\right)\\
\left\langle \frac{\ddot{p}(r(t))}{2}g^2\right\rangle&=&\int P(r,t)\frac{g^2}{2}\frac{d^2}{dr^2}\delta(r-R)dr\\
&=&\int \delta(r-R)\frac{d^2}{dr^2}P(r,t)\frac{g^2}{2}dr\\
&=&\frac{g^2}{2}\frac{\partial}{\partial R^2}P(R,t).
\end{eqnarray}
Finally, substituting the above expression into Eq.(16) yields the Fokker-Planck equation shown below
\begin{equation}
\frac{\partial}{\partial t}P(R,t)=-\frac{\partial}{\partial  R}\left(P(R,t)f(R,t)\right)
+\frac{g^2}{2}\frac{\partial}{\partial R^2}P(R,t).
\end{equation}
\\
We can recover the original Langevin equation by changing variables as follows
\begin{eqnarray}
R&=&R_\alpha\\
f(R,t)&=&0\\
g^2&=&\frac{2k_BT}{\zeta}=2D,
\end{eqnarray}
then, the Fokker-Planck equation takes the usual form of a diffusion equation
\begin{equation}
\frac{\partial}{\partial t}P(R_\alpha,t)=
D\frac{\partial}{\partial R_\alpha^2}P(R_\alpha,t).
\end{equation}
Solving this with an initial condition $P(R_\alpha,t)=\delta(R_\alpha)$, yields 
\begin{eqnarray}
  P(R_\alpha,t)&=&\frac{1}{\sqrt{4\pi Dt}}\exp\left[-\frac{R_\alpha^2}{4Dt}\right]\\
  &=&\frac{1}{\sqrt{2\pi \sigma^2}}\exp\left[-\frac{R_\alpha^2}{2\sigma^2}\right]
\end{eqnarray}
with $\sigma^2=\frac{2k_BT t}{\zeta}=2Dt$.
This is equivalent to Eq.(G1)-(G3).
\end{document}
